\chapter{Összegzés, kitekintés}
\label{ch:sum}

A dolgozatom keretein belül megvizsgáltam a zenei információk kinyerése (MIR) tudományterület feladatait, megismerkedtem az automatikus hangszerfelismerés problémájával, ennek megközelítéseivel. Felkutattam különböző gépi tanulási és mély tanulási megoldásokat, tanító adathalmazokat.

Bemutattam egy CNN architektúrát, amellyel három különböző bemeneti reprezentáción kísérleteztem. A kísérletek konklúziójaként az látszik, hogy míg a bemutatott mély tanulási architektúra a már előre tanított, kisebb dimenziójú reprezentáción nagyjából egyforma teljesítményt nyújt, addig a nagyobb felbontású reprezentációkon már jobban teljesít, mint egy hagyományos gép tanulási megoldás.

A jövőben több módon is tovább lehetne optimalizálni a bemutatott modellt. A tapasztalataim alapján úgy gondolom, hogy a legnagyobb javulást az adathalmaz bővítése eredményezné. Erre azonban nem biztos, hogy van lehetőség, így én a következő továbbfejlesztéseket fontolnám meg:
\begin{itemize}
 \item Undersampling helyett oversampling, vagy data augmentation technikát alkalmazni az adatok előfeldolgozásakor
 \item A jelenlegi CNN architektúrát tovább mélyíteni, vagy helyette RNN architektúrát implementálni
 \item Az early stopping technika mellé model checkpointokat is bevezetni.
\end{itemize}

A felsoroltakon felül pedig érdemes lehet a melspectogram, MFCC és esetleg további reprezentációk mély tanulásra vonatkozó hatékonyságbeli különbségeit is tovább elemezni.