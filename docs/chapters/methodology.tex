\chapter{Módszertan} 
\label{ch:methodology}

Ebben a fejezetben az alkalmazott módszertant mutatom be. Először ismertetem az alkalmazott előfeldolgozási módszereket. Ezután a tanító modellek architektúráját mutatom be. Végül pedig szót ejtek az ismertetett módszerek implementációjáról.

\section{Bemeneti adatok, előfeldolgozás}

Az OpenMIC adathalmazzal két formában kapjuk meg a bemeneti adatokat. Egyrészt elérhetőek a nyers hanganyagok .ogg formátumban. Ezen kívül rendelkezésre állt a VGGish nevű reprezentáció is, amelyről a korábbi \ref{subsec:VGGish} fejezetben írtam. A bemeneti adatokat tanítás előtt azonban néhány előfeldolgozási folyamaton vittem végig részben a megvalósíthatóság, részben pedig a performancianövelés érdekében. Ezekről írok a következő alfejezekben.

\subsection{Alternatív reprezentációk kinyerése}

A .ogg fájlokból beolvasott nyers audioval való tanítás sajnos nem volt lehetséges, mivel ez egy memória szempontjából igen költséges reprezentáció, amihez a hardver eszközöm kapacitása kevésnek bizonyult. Helyette a dolgozat korábbi fejezetében bemutatott melspectogram és MFCC reprezentációkat használtam fel. Mindkét reprezentációt a beolvasott nyers hullámforma reprezentációból nyertem ki az elméleti háttér fejezetben leírt módon.

\subsection{Adathalmazok normalizálása}

Az adatok normalizálása a tanulási folyamat gyorsítására szolgál. A normalizálást úgy érjük el, hogy az adatokat arányosan a 0 és 1 értékek közé transzformáljuk. Erre egy szokásos megoldás:
\begin{equation}
x(_i) := (x(_i) - x_{\text{min}}) \/ (x_{\text{max}} - x_{\text{min}})
\end{equation}

A normalizált adathalmon a gradiens általában hamarabb csökkenthető, ezért a tanítási folyamat rövidebb lehet. \cite{LeCun2012}

\subsection{Osztályok kiegyensúlyozása}

Tanító modelleknél gyakori probléma, hogy a tanulni kívánt osztályok egy része alul van reprezentálva. Dolgozatom multi-label osztályozása esetén két osztályról beszélhetünk: igaz (jelen van az adott hangszer) és hamis (nincs jelen az adott hangszer). Mivel sok esetben a hamis értékek jelentős többségben voltak, - egyes hangszerek esetén az összes adat közel 80\%-át is kitette - ezért a modell jó pontosságot tudott elérni csupán ''hamis'' predikciókkal.

Ezt elkerülendő, három megoldás jöhetett szóba:
\begin{itemize}
 \item Alulmintavételezés (Undersampling) - a többségi osztályokból véletlenszerűen eltávolítunk annyi elemet, hogy az osztályok kiegyensúlyozottá váljanak.
 \item Túlmintavételezés (Oversampling) - a kisebbségi osztályok véletlenszerű elemeit duplikáljunk addig, amíg az osztályok kiegyensúlyozottá válnak.
 \item Adat augmentáció (Data augmentation) - hasonlóan az oversampling technikához a kisebbségi osztály elemeit bővítjük. Ebben az esetben a meglévő elemek transzformálásával mesterségesen hozunk létre új elemeket (pl. képeknél forgatás). \cite{imbalanced}
\end{itemize}

Dolgozatom keretében az undersampling módszert alkalmaztam.

\subsection{Tanító-, tesztelő- és validáló halmaz kialakítása}

Tanító modellek alkalmazásakor fontos, hogy legalább kettő, de inkább három független adathalmazzal rendelkezzünk: 
\begin{itemize}
 \item Tanító adathalmaz (Train Set) - ebből a halmazból tanul és ezt a halmazt látja a modellünk.
 \item Validáló adathalmaz (Validation Set) - a tanítási lépések (epoch-ok) között ezen a tanító halmaztól független adathalmazon kiértékeljük modellünk működését. Ezáltal visszajelzést tudunk adni a tanítási folyamat felé és ha szükséges, változtathatunk a hiperparamétereken. Modellünk tehát ezt a halmazt időnként látja, de nem ebből tanul.
 \item Teszt adathalmaz (Test Set) - a tanítási folyamat végén ezen a halmazon értékeljük ki modellünk teljesítményét. Modellünk ezt a halmazt a tanítás alatt egyáltalán nem látja és ezáltal nem is tud tanulni belőle. Ezzel a függetlenséggel biztosítjuk a kiértékelés torzítatlanságát. \cite{traintestvalid}
\end{itemize}

Bevett szokás, hogy e három halmazt körülbelül 60\%-20\%-20\% arányban osszuk fel a tanító halmaz javára. \cite{traintestvalid} Dolgozatom során én is ezt az elvet követtem.

\section{Architektúra}

A következő alfejezetekben a különböző tanító modell kísérletek architektúráját mutatom be.

\subsection{Hagyományos Machine Learning - Modeling Baseline}
//TODO felhasznált deep learning architektúra(?)

+ alternatív reprezentációkkal is
Az OpenMIC dataset megalkotói bemutattak egy példa modelt, amely egy véletlen erdő osztályozó (Random Forest Classifier) modell. Ez egy hagyományos gépi tanulási modell. Az osztályozó 100 eldöntési fából áll, a maximális mélysége 8, 

\subsection{VGGish Embedding Downstream CNN}


\subsection{Deep CNN}


\section{Megvalósítás}

//TODO implementáció részletei

python keras tensorflow numpy scipy librosa 