\chapter{Adathalmaz}
\label{ch:dataset}

Ebben a fejezetben a munkám kapcsán felkutatott és alkalmazott adathalmazokról lesz szó. Egy deep learning megoldás tervezésének első lépéseként érdemes egy alkalmas kiinduló adathalmazt kiválasztani. Ezt aztán a modell tanítására és tesztelésre használjuk.

\section{Kiválasztási szempontok}

//TODO ismir dataset gyűjtemény
pl. többszólam, ingyenesen elérhető, szerteágazó (not biased), stb gyenge címkézés

\section{Philharmonia Orchestra}

Kutatásom első fázisában a Philharmonia Zenekar ingyenesen elérhető hangminta könyvtárát használtam fel. Ebben egyszólamú mintákat találunk. A minták a főkönyvtáron belül a bennük megszólaló hangszer nevével megegyező könytárban találhatóak, ez biztosítja a címkéket. 

//TODO tulajdonságai

\section{OpenMIC}

A többszólamúság bevezetését a kutatásomban az OpenMIC \cite{humphrey2018openmic} adathalmaz felhasználásával értem el.  
//TODO  openmic cikk alapján