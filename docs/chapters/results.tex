\chapter{Kísérletek, eredmények} 
\label{ch:results}

Lorem ipsum

\section{Mérőszámok}

//TODO: https://towardsdatascience.com/whats-the-deal-with-accuracy-precision-recall-and-f1-f5d8b4db1021 \cite{fscore}

könyv alapján
hogyan tudjuk megmérni a modell teljesítményét, pontosság, tanítás ideje, stb...

A következő metrikákat vizsgáltuk:

\begin{itemize}
\item Pontosság (accuraccy) - a modell az adott lépésben a bemeneti adatok hány százalékára adott helyes kimenetet?
\item Veszteség (loss) - a veszteségfüggvény eredménye. A modell predikcióinak a valóságtól való eltérését összeadva kapjuk meg.
\item Precizitás (precision) - 
\item Felidézés (recall) - 
\item F1 érték (F1 score) - ..... Ennek a súlyozott átlaga a mérvadó.


F1 érték hangszerenként, ezek átlaga, legkisebb és legnagyobb közti különbség
\end{itemize}

\section{Eredmények}


\begin{tikzpicture}
    \begin{axis}[
        width  = \textwidth,
        height = 10cm,
        major x tick style = transparent,
        ybar=0pt,
        bar width=4pt,
        ymajorgrids = true,
        ylabel = {macro f-score},
        symbolic x coords={Harmónika,Bendzsó,Basszusgitár,Cselló,Klarinét,Cintányér,Dob,Furulya,Gitár,Melodikus ütőhangszer,Mandolin,Orgona,Zongora,Szaxofon,Szintetizátor,Harsona,Trombita,Ukulele,Hegedű,Ének},
        xtick = data,
        x tick label style={rotate=90},
        scaled y ticks = false,
legend cell align=left,legend style={
                at={(1,1.05)},
                anchor=south east,
                column sep=1ex
        }
    ]
        \addplot[style={bblue,fill=bblue,mark=none}]
            coordinates {(Harmónika, 0.68) (Bendzsó,0.72) (Basszusgitár,0.75)  (Cselló,0.79)  (Klarinét,0.52)  (Cintányér,0.93)  (Dob,0.75)  (Furulya,0.89)  (Gitár,0.97)  (Melodikus ütőhangszer,0.77)  (Mandolin,0.70)  (Orgona,0.60) (Zongora,0.93) (Szaxofon,0.83) (Szintetizátor,0.94) (Harsona,0.75) (Trombita,0.76) (Ukulele,0.72) (Hegedű,0.83) (Ének,0.93)};

        \addplot[style={rred,fill=rred,mark=none}]
            coordinates {(Harmónika, 0.76) (Bendzsó,0.74) (Basszusgitár,0.66) (Cselló,0.73) (Klarinét,0.52)  (Cintányér,0.88) (Dob,0.93) (Furulya,0.66) (Gitár,0.95)  (Melodikus ütőhangszer,0.79)  (Mandolin,0.64)  (Orgona,0.77) (Zongora,0.91) (Szaxofon,0.67) (Szintetizátor,0.89) (Harsona,0.77) (Trombita,0.71) (Ukulele,0.64) (Hegedű,0.76) (Ének,0.81)};


        \legend{ML VGGish ,DL VGGish}
    \end{axis}
\end{tikzpicture}

\begin{tikzpicture}
    \begin{axis}[
        width  = \textwidth,
        height = 10cm,
        major x tick style = transparent,
        ybar=0pt,
        bar width=4pt,
        ymajorgrids = true,
        ylabel = {macro f-score},
        symbolic x coords={Harmónika,Bendzsó,Basszusgitár,Cselló,Klarinét,Cintányér,Dob,Furulya,Gitár,Melodikus ütőhangszer,Mandolin,Orgona,Zongora,Szaxofon,Szintetizátor,Harsona,Trombita,Ukulele,Hegedű,Ének},
        xtick = data,
        x tick label style={rotate=90},
        scaled y ticks = false,
legend cell align=left,legend style={
                at={(1,1.05)},
                anchor=south east,
                column sep=1ex
        }
    ]
        \addplot[style={ppurple,fill=ppurple,mark=none}]
            coordinates {(Harmónika,0.45) (Bendzsó,0.39) (Basszusgitár,0.42) (Cselló,0.63)  (Klarinét,0.43)  (Cintányér,0.80)  (Dob,0.82)  (Furulya,0.41)  (Gitár,0.47)  (Melodikus ütőhangszer,0.52)  (Mandolin,0.39)  (Orgona,0.59) (Zongora,0.84) (Szaxofon,0.59) (Szintetizátor,0.51) (Harsona,0.47) (Trombita,0.59) (Ukulele,0.43) (Hegedű,0.57) (Ének,0.82) };

        \addplot[style={ggreen,fill=ggreen,mark=none}]
            coordinates {(Harmónika,0.46) (Bendzsó,0.46) (Basszusgitár,0.74) (Cselló,0.65)  (Klarinét,0.61)  (Cintányér,0.80)  (Dob,0.83)  (Furulya,0.54)  (Gitár,0.45)  (Melodikus ütőhangszer,0.57)  (Mandolin,0.57)  (Orgona,0.59) (Zongora,0.84) (Szaxofon,0.55) (Szintetizátor,0.73) (Harsona,0.54) (Trombita,0.44) (Ukulele,0.50) (Hegedű,0.68) (Ének,0.68) };
 	
        \legend{ML Mel,DL Mel}
    \end{axis}
\end{tikzpicture}




\subsection{Modeling Baseline}


\subsection{Modeling Baseline alternatív reprezentációkkal}

\subsection{VGGish Embedding Downstream CNN}

\subsection{Deep CNN}

\section{Összehasonlítások}

\subsection{saját próbálkozásaim} 

\subsection{Más ismert munkákkal}