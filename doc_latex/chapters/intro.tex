\chapter{Bevezetés} % Introduction
\label{ch:intro}


\section{Motiváció}

Az elmúlt években egyre felkapottabb kutatási területté váltak a mély neuronhálós rendszerek. Ezen belül a zenei információk kinyerésének kutatása is jelentős teret hódított magának. 

\section{A dolgozat felépítése}

Dolgozatomban tehát az eddigi kapcsolódó kutatásokat, illetve saját munkám eredményét dolgozom fel. A következő alfejezetben felsorolom az általam relevánsnak tartott, a State-of-the-Art-hoz vezető kutatásokat. 

A második fejezetben betekintést adok a téma elméleti hátterébe. Először kifejtem a zenével kapcsolatos főbb fogalmakat, bemutatom fontosabb tulajdonságait, reprezentációit. Kitérek a kapcsolódó kutatási terület bemutatására is. Ezután bevezetem a gépi tanulás és a mély tanulás fogalmát. 

A harmadik fejezetben a módszertanról ejtek szót. Itt kifejtésre kerülnek az adat előfeldolgozási módszerei, az általam bemutaott mély tanulási architektúrák, illetve ezek megvalósításai. A negyedik fejezet az adathalmazokról fog szólni. Itt először felsorolom az adathalmazok kiválasztásának szempontjait, majd minden felhasznált adathalmaznak leírom a főbb jellemzőit. 

Az ötödik fejezetben részletezem az általam végzett kisérleteket és ezek eredményeit. Eután ezeket összevetem egymással, illetve a releváns State-of-the-Art kutatásokkal. A hatodik fejezetben összegzem a leírtakat és kutatásom továbbgondolására felvázolok néhány ötletet.


\section{Kapcsolódó munkák}

Cite author  \citeauthor{li2015automatic} and cite \cite{han2016deep}. 